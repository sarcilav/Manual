\documentclass[10pt,letterpaper,twocolumn,twosided]{article}
%\documentclass[10pt,letterpaper]{article}
% ---------------------------------------------------------------
\usepackage[utf8]{inputenc}
\usepackage[spanish]{babel}
\usepackage{listings}
\usepackage[usenames,dvipsnames]{color}
\usepackage{amsmath}
\usepackage{verbatim}
\usepackage{hyperref}
\usepackage{color}
\usepackage{geometry}
\usepackage{fancyhdr}
\fancyhf{} % clear all header and footers
\renewcommand{\headrulewidth}{0pt} % remove the header rule
\rhead{\thepage}
%
%lfoot{\thepage} % puts it on the left side instead
%
% or if your document is 2 sided, and you want even and odd placement of the number
%facncyfoot[LE,RO]{\thepage} % Left side on Even pages; Right side on Odd pages
%
\pagestyle{fancy}


%Poner la página en landscape
\geometry{verbose,landscape,letterpaper,tmargin=2cm,bmargin=1.5cm,lmargin=1cm,rmargin=1cm}


\newcommand{\codigofuente}[1]{
\verbatiminput{#1}
\dotfill
}
\setlength{\columnsep}{0.25in}
\setlength{\columnseprule}{1px}

% ---------------------------------------------------------------
\begin{document}
% ---------------------------------------------------------------
\title{Resumen de algoritmos para torneos de programación}
\author{Sebastián Arcila Valenzuela}

\date{\today}
\maketitle
% ---------------------------------------------------------------

% ---------------------------------------------------------------
\lstloadlanguages{C++}
% ---------------------------------------------------------------
\section{Plantilla}
\codigofuente{./src/template.cpp}
% ---------------------------------------------------------------
\section{Teoría de números}
% ---------------------------------------------------------------
\subsection{Big mod}
%% Generator: GNU source-highlight, by Lorenzo Bettini, http://www.gnu.org/software/src-highlite

{\ttfamily \raggedright {
\noindent
\mbox{}\textit{\textcolor{Brown}{//retorna\ (b\textasciicircum{}p)mod(m)}} \\
\mbox{}\textit{\textcolor{Brown}{//\ 0\ $<$=\ b,p\ $<$=\ 2147483647}} \\
\mbox{}\textit{\textcolor{Brown}{//\ 1\ $<$=\ m\ $<$=\ 46340}} \\
\mbox{}\textcolor{ForestGreen}{long}\ \textbf{\textcolor{Black}{f}}\textcolor{BrickRed}{(}\textcolor{ForestGreen}{long}\ b\textcolor{BrickRed}{,}\ \textcolor{ForestGreen}{long}\ p\textcolor{BrickRed}{,}\ \textcolor{ForestGreen}{long}\ m\textcolor{BrickRed}{)}\textcolor{Red}{\{} \\
\mbox{}\ \ \textcolor{ForestGreen}{long}\ mask\ \textcolor{BrickRed}{=}\ \textcolor{Purple}{1}\textcolor{BrickRed}{;} \\
\mbox{}\ \ \textcolor{ForestGreen}{long}\ pow2\ \textcolor{BrickRed}{=}\ b\ \textcolor{BrickRed}{\%}\ m\textcolor{BrickRed}{;} \\
\mbox{}\ \ \textcolor{ForestGreen}{long}\ r\ \textcolor{BrickRed}{=}\ \textcolor{Purple}{1}\textcolor{BrickRed}{;} \\
\mbox{} \\
\mbox{}\ \ \textbf{\textcolor{Blue}{while}}\ \textcolor{BrickRed}{(}mask\textcolor{BrickRed}{)}\textcolor{Red}{\{} \\
\mbox{}\ \ \ \ \textbf{\textcolor{Blue}{if}}\ \textcolor{BrickRed}{(}p\ \textcolor{BrickRed}{\&}\ mask\textcolor{BrickRed}{)} \\
\mbox{}\ \ \ \ \ \ r\ \textcolor{BrickRed}{=}\ \textcolor{BrickRed}{(}r\ \textcolor{BrickRed}{*}\ pow2\textcolor{BrickRed}{)}\ \textcolor{BrickRed}{\%}\ m\textcolor{BrickRed}{;} \\
\mbox{}\ \ \ \ pow2\ \textcolor{BrickRed}{=}\ \textcolor{BrickRed}{(}pow2\textcolor{BrickRed}{*}pow2\textcolor{BrickRed}{)}\ \textcolor{BrickRed}{\%}\ m\textcolor{BrickRed}{;} \\
\mbox{}\ \ \ \ mask\ \textcolor{BrickRed}{$<$$<$=}\ \textcolor{Purple}{1}\textcolor{BrickRed}{;} \\
\mbox{}\ \ \textcolor{Red}{\}} \\
\mbox{}\ \ \textbf{\textcolor{Blue}{return}}\ r\textcolor{BrickRed}{;} \\
\mbox{}\textcolor{Red}{\}} \\

} \normalfont\normalsize
%.tex
\codigofuente{./../teoria_numeros/bigMod.cpp}

\subsection{Phi de Euler}
\codigofuente{./../teoria_numeros/phiEu.cpp}

\subsection{GCD Extendido}
\codigofuente{./../teoria_numeros/extended_gcd.cpp}

\subsection{Fibo}
\codigofuente{./../teoria_numeros/fib.cpp}

\subsection{Criba de Eratóstenes}
\codigofuente{./src/number_theory/criba.cpp}

\subsection{Factores Primos}
\codigofuente{./../teoria_numeros/prim_factors.cpp}

\section{Combinatoria}
\subsection{Cuadro resumen}
Fórmulas para combinaciones y permutaciones:
\begin{center}
  \renewcommand{\arraystretch}{2} %Multiplica la altura de cada fila de la tabla por 2
  % Si quiero aumentar el tamaño de una fila en particular insertar \rule{0cm}{1cm} en esa fila.
  \begin{tabular}{| c | c | c |}
    \hline
    \textit{Tipo} & \textit{¿Se permite la repetición?} & \textit{Fórmula} \\ [1.5ex]
    \hline\hline

    $r$-permutaciones & No & $ \displaystyle\frac{n!}{(n-r)!} $ \\ [1.5ex]
    \hline
    $r$-combinaciones & No & $ \displaystyle\frac{n!}{r!(n-r)!} $ \\  [1.5ex]
    \hline
    $r$-permutaciones & Sí & $ \displaystyle n^{r} $ \\
    \hline
    $r$-combinaciones & Sí & $ \displaystyle\frac{(n+r-1)!}{r!(n-1)!} $ \\ [1.5ex]
    \hline
  \end{tabular}
  \renewcommand{\arraystretch}{1}
\end{center}

\subsection{Combinaciones, coeficientes binomiales, triángulo de Pascal}
\emph{Complejidad:} $ O(n^2) $ \\
$$ {n \choose k} = \left\{
  \begin{array}{c l}
    1 & k = 0\\
    1 & n = k\\
    \displaystyle {n - 1 \choose k - 1} + {n - 1 \choose k} & \mbox{en otro caso}\\
  \end{array}
\right.
$$

\codigofuente{./src/combinatoria/pascal_triangle.cpp}

\bigskip
\textbf{Nota:} $ \displaystyle {n \choose k }  $ está indefinido en el código anterior si $ n > k$. ¡La tabla puede estar llena con cualquier basura del compilador!

\subsection{Permutaciones con elementos indistinguibles}
El número de permutaciones \underline{diferentes} de $n$ objetos, donde hay $n_{1}$ objetos indistinguibles de tipo 1,
$n_{2}$ objetos indistinguibles de tipo 2, ..., y $n_{k}$ objetos indistinguibles de tipo $k$, es
$$
\frac{n!}{n_{1}!n_{2}! \cdots n_{k}!}
$$
\textbf{Ejemplo:} Con las letras de la palabra \texttt{PROGRAMAR} se pueden formar $ \displaystyle \frac{9!}{2! \cdot 3!} =
30240 $ permutaciones \underline{diferentes}.
\subsection{Desordenes, desarreglos o permutaciones completas}

Un desarreglo es una permutación donde ningún elemento $i$ está en la
posición $i$-ésima. Por ejemplo, \textit{4213} es un desarreglo de 4 elementos pero
\textit{3241} no lo es porque el 2 aparece en la posición 2.

Sea $D_{n}$ el número de desarreglos de $n$ elementos, entonces:
$$ {D_{n}} = \left\{
  \begin{array}{c l}
    1 & n = 0\\
    0 & n = 1\\
    (n-1)(D_{n-1} + D_{n-2}) & n \geq 2\\
  \end{array}
\right.
$$
Usando el principio de inclusión-exclusión, también se puede encontrar la fórmula
$$
D_{n} = n!\left [ 1 - \frac{1}{1!} + \frac{1}{2!} - \frac{1}{3!} + \cdots + (-1)^{n}\frac{1}{n!} \right ]
= n! \sum_{i=0}^{n} \frac{(-1)^{i}}{i!}
$$

\section{Grafos}
\subsection{Topological Sort(BFS)}
\codigofuente{./../graphs/topological_sort.cpp}
\subsection{Longest Path in DAG}
\codigofuente{./../graphs/longest_path_in_dag.cpp}
\subsection{Algoritmo de Dijkstra}
El peso de todas las aristas debe ser no negativo.
\\
\codigofuente{./src/grafos/dijkstra.cpp}

\subsection{Minimum spanning tree: Algoritmo de Kruskal + Union-Find}
\codigofuente{./src/grafos/kruskal.cpp}

\subsection{Algoritmo de Floyd-Warshall}
\codigofuente{./src/grafos/floyd.cpp}

\subsection{Algoritmo de Bellman-Ford}
Si no hay ciclos de coste negativo, encuentra la distancia más corta
entre un nodo y todos los demás. Si sí hay, permite saberlo. \\ El
coste de las aristas \underline{sí} puede ser negativo
(\emph{Debería}, si no es así se puede usar Dijsktra o Floyd).
\codigofuente{./src/grafos/bellman.cpp}

\subsection{Puntos de articulación}
\codigofuente{./src/grafos/puntos_articulacion.cpp}

\subsection{Máximo flujo: Método de Ford-Fulkerson, algoritmo de Edmonds-Karp}
El algoritmo de Edmonds-Karp es una modificación al método de Ford-Fulkerson. Este último
utiliza DFS para hallar un camino de aumentación, pero la sugerencia de Edmonds-Karp
es utilizar BFS que lo hace más eficiente en algunos grafos.
\medskip

\codigofuente{./src/grafos/ford_fulkerson.cpp}

\subsection{Máximo flujo para grafos dispersos usando Ford-Fulkerson}
\codigofuente{./src/grafos/ford_fulkerson_sparse.cpp}

\subsection{Máximo flujo para grafos disperos usando algoritmo de Dinic}
\codigofuente{./src/grafos/dinic.cpp}

\subsection{Máximo flujo mínimo costo}
Implementación de Misof:
\codigofuente{./src/grafos/mincostmaxflow.cpp}

\subsection{Componentes fuertemente conexas: Algoritmo de Tarjan}
\label{tarjan}
\codigofuente{./src/grafos/tarjan.cpp}

\subsection{2-Satisfiability}
Dada una ecuación lógica de conjunciones de disyunciones de 2 términos, se pretente decidir si existen valores de verdad que puedan asignarse a las variables para hacer cierta la ecuación. \\
Por ejemplo, $(b_1 \vee \neg b_2) \wedge (b_2 \vee b_3) \wedge (\neg b_1 \vee \neg b_2) $ es verdadero cuando $b_1$ y $b_3$ son verdaderos y $b_2$ es falso. \\
\textbf{Solución:} Se sabe que $(p \rightarrow q) \leftrightarrow (\neg p \vee q)$. Entonces se traduce cada disyunción en una implicación y se crea un grafo donde los nodos son cada variable y su negación. Cada implicación es una arista en este grafo. Existe solución si nunca se cumple que una variable y su negación están en la misma componenete fuertemente conexa (Se usa el algoritmo de Tarjan, \ref{tarjan}).

\section{Programación dinámica}
\subsection{Kadane}
\codigofuente{./../dp/kadane.cpp}
\subsection{LIS}
\codigofuente{./../dp/lis.cpp}
\subsection{Longest common subsequence}
\codigofuente{./src/dp/lcs.cpp}
\subsection{Partición de troncos}
Este problema es similar al problema de \textit{Matrix Chain Multiplication}. Se tiene
un tronco de longitud $n$, y $m$ puntos de corte en el tronco. Se puede hacer un corte a la vez,
cuyo costo es igual a la longitud del tronco. ¿Cuál es el mínimo costo para partir todo el tronco
en pedacitos individuales?
\\
\medskip
\textbf{Ejemplo:} Se tiene un tronco de longitud 10. Los puntos de corte son 2, 4, y 7. El mínimo
costo para partirlo es 20, y se obtiene así:
\begin{itemize}
\item Partir el tronco (0, 10) por 4. Vale 10 y quedan los troncos (0, 4) y (4, 10).
\item Partir el tronco (0, 4) por 2. Vale 4 y quedan los troncos (0, 2), (2, 4) y (4, 10).
\item No hay que partir el tronco (0, 2).
\item No hay que partir el tronco (2, 4).
\item Partir el tronco (4, 10) por 7. Vale 6 y quedan los troncos (4, 7) y (7, 10).
\item No hay que partir el tronco (4, 7).
\item No hay que partir el tronco (7, 10).
\item El costo total es $10+4+6 = 20$.
\end{itemize}

\medskip
El algoritmo es $O(n^3)$, pero optimizable a $O(n^2)$ con una tabla adicional:
\codigofuente{./src/dp/particion_troncos.cpp}

\section{Geometría}
\subsection{Cabe, Area y Centroide}
\codigofuente{./../geometria/geometria.cpp}

\subsection{Convex hull: Graham Scan}
\emph{Complejidad:} $ O(n \log_{2}{n}) $
\codigofuente{./src/geometria/grahamscan.cpp}

\subsection{Mínima distancia entre un punto y un segmento}
\codigofuente{./src/geometria/distance_point_to_segment.cpp}

\subsection{Mínima distancia entre un punto y una recta}
\codigofuente{./src/geometria/distance_point_to_line.cpp}

\subsection{Determinar si un polígono es convexo}
\codigofuente{./src/geometria/is_convex_polygon.cpp}

\subsection{Determinar si un punto está dentro de un polígono convexo}
\codigofuente{./src/geometria/is_inside_convex_polygon.cpp}

\subsection{Determinar si un punto está dentro de un polígono cualquiera}
\small
\textbf{Field-testing:}
\begin{itemize}
\item \emph{TopCoder} -  SRM 187 - Division 2 Hard - PointInPolygon
\end{itemize}
\codigofuente{./src/geometria/is_inside_concave_polygon.cpp}

\subsection{Intersección de dos rectas}
\codigofuente{./src/geometria/line_line_intersection.cpp}

\subsection{Intersección de dos segmentos de recta}
\codigofuente{./src/geometria/segment_segment_intersection.cpp}
% ---------------------------------------------------------------
\section{Estructuras de datos}
\subsection{RMQ}
\codigofuente{./../lca/lubenica.cpp}
\subsection{Árboles de Fenwick ó Binary indexed trees}

Se tiene un arreglo $\{a_0, a_1, \cdots, a_{n-1}\}$. Los árboles
de Fenwick permiten encontrar $ \displaystyle \sum_{k=i}^{j} a_k $ en orden $O(\log_{2}{n})$ para parejas de $(i, j)$ con $i \leq j$. De la misma manera, permiten sumarle una cantidad a un $a_i$ también en tiempo $O(log_{2}{n})$.
\medskip
\codigofuente{./../bit/simple-binary-indexed-tree.cc}

\subsection{Segment tree}
\codigofuente{./../segment-tree/segment.cpp}

\section{String Matching}
\subsection{KMP}
\codigofuente{./../string_matching/kmp.cpp}
\subsection{Suffix Arrays}
\codigofuente{./../string_matching/suffix_arrays.cpp}


\end{document}